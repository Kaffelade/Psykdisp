% Humanistisk psykologi
% 8 dec
% Anvendt Psykolog af Martin Levander et.al.
% Copyright Maria Scott


\section{Humanistisk psykologi}

I modsætning til psykoanalyse og adfærdsterapi/behaviorismen har
humanistisk psykolog flere facetter og flere psykologer med forskellige
holdninger. Humanistisk psykologi kom til i 40'erne som den tredje
bevægelse.

\subsection{Carl Rogers (1902-1987)}

Mennesket er en organisme - en fysisk og psykisk helhed. For at kunne
forstå et menneske, må man se verden gennem dennes øjne. Man lægger mærke 
til hvad, der interesserer en, resten er ubevidste oplevelser. Børns 
bevidsthed er mindre styret, de lægger mærke til flere ting. Med 
personlighedens udvikling, sorteres flere ting fra.

Oplevelser (tanker og følelser) kan blive forstyrret, hvis man har en 
modstridende selvopfattelse. Her mangler kongruens - man er ubevidst om 
følelsen, fordi man ikke anerkender den. Eksempel med præsten, der føler 
sig tiltrukket af en ung pige, men ikke anerkender det, så han ser ned på 
hende i stedet. Også 'Miss Har', som anerkender sin fars negative sider, 
fordi hun har sat sig for, at hun hader ham. Under terapi anerkender hun 
også hans gode egenskaber. Her er manglende selverkendelse (Freud ville 
kalde det forsvarsmekanismer).

Forskel mellem idealselv og selv: selvet er hvordan vi opfatter os selv 
og idealselvet er, hvordan vi gerne vil være. Uoverensstemmelser fører 
til utilfredshed.

Manglende kongruens kan stamme fra man er barn. Børn skal opleve 
betingelsesløs kærlighed, altså ikke 'hvis du ikke opfører dig pænt, 
så elsker jeg dig ikke'. Ikke at forveksle med at man ikke skal sætte 
grænser. Seksualitet er et vigtigt område, hvor man let kan skade en person. 
Hvis det i familien er skamfuldt, kan man måske ikke nyde sex, når man er 
af den opfattelse at det er forkert. Man anerkender ikke sin egen lyst.

\subsubsection{Klientcentreret terapi}

Afdramatiseret terapi, der kaldes vejledning og patient kaldet klient. Skal 
mindske afstanden mellem selv og idealselv. Selvet bringes nærmere ved 
at blive bevidst om og anerkende 'forbudte' eller skjulte følelser, mens 
idealselvet justeres til et mere realistisk ideal. Terapeuten skulle i 
følge Rogers i høj grad være ligestillet med og vise sig  engageret 
i klienten i modsætning til Freud, hvis forestilling var, at den 
psykoanalytiske terapeut skulle tage sig i agt for at optræde 
følelsesmæssigt.
følelser
